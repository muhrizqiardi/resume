\documentclass{resume} % Use the custom resume.cls style

\usepackage[left=0.4 in,top=0.4in,right=0.4 in,bottom=0.4in]{geometry} % Document margins
\newcommand{\tab}[1]{\hspace{.2667\textwidth}\rlap{#1}} 
\newcommand{\itab}[1]{\hspace{0em}\rlap{#1}}
\name{Muhammad Rizqi Ardiansyah} % Your name
% You can merge both of these into a single line, if you do not have a website.
\address{+62 812 1002 9357 \\ Surabaya, East Java, Indonesia} 
\address{\href{mailto:muhrizqiardi@gmail.com}{muhrizqiardi@gmail.com} \\ \href{https://www.linkedin.com/in/muhrizqiardi/}{linkedin.com/in/muhrizqiardi/} \\ \href{https://muhrizqiardi.vercel.app}{muhrizqiardi.vercel.app}}  %

\begin{document}

%----------------------------------------------------------------------------------------
%	OBJECTIVE
%----------------------------------------------------------------------------------------

\begin{rSection}{ABOUT}

{A front end developer passionate about technology. Fast-learning and always ready of what's to come in this rapidly-moving industry.}


\end{rSection}
%----------------------------------------------------------------------------------------
%	EDUCATION SECTION
%----------------------------------------------------------------------------------------

\begin{rSection}{Education}

{\bf State University of Surabaya (UNESA)}, Informatics Engineering \hfill {2022 - Ongoing}

\end{rSection}

%----------------------------------------------------------------------------------------
% TECHINICAL STRENGTHS	
%----------------------------------------------------------------------------------------
\begin{rSection}{SKILLS}

\begin{tabular}{ @{} >{\bfseries}l @{\hspace{6ex}} l }
HTML, CSS \\
JavaScript, TypeScript \\
React \\
Next.js \\
TailwindCSS \\
Python \\
Git version control system \\
\end{tabular}\\
\end{rSection}

\begin{rSection}{EXPERIENCE}

\textbf{Freelance Frontend Engineer} \hfill Dec 2021 - Jun 2022\\
Rumah Coding Cerdas 
 \begin{itemize}
    \itemsep -3pt {} 
     \item Implement a single-page application (SPA) website using React based on a provided design and  requirements
     \item Write a unit tests for the written code
    \item Collaborate with the team to implement new features and solve issues
 \end{itemize}
 
% \textbf{Role Name} \hfill Jan 2017 - Jan 2019\\
% Company Name \hfill \textit{San Francisco, CA}
%  \begin{itemize}
%     \itemsep -3pt {} 
%      \item Achieved X\% growth for XYZ using A, B, and C skills.
%      \item Led XYZ which led to X\% of improvement in ABC
%     \item Developed XYZ that did A, B, and C using X, Y, and Z. 
%  \end{itemize}

\end{rSection} 

%----------------------------------------------------------------------------------------
%	WORK EXPERIENCE SECTION
%----------------------------------------------------------------------------------------

\begin{rSection}{PROJECTS}
\vspace{-1.25em}
\item \textbf{LABCO} {A lab management system made for final project in a Software Engineering class with my teammates. The app was made using Next.js framework, both the frontend and backend. Used TailwindCSS for the styling, and MongoDB for the non-relational database.\href{https://labco.vercel.app/}{(labco.vercel.app)}}
\item \textbf{SpaceWork} {A remote job listing web app, made for final project in Web Programming class with my team. The front-end side was made using React, with TailwindCSS as the styling. The back-end side was made using Fastify. The database used in this project are PostgreSQL, hosted from Supabase.}
\item \textbf{Booktracker} {A mobile app to track book progress, made for project in Mobile Programming class with my team. The app was made using Flutter. It stores data using SQLite, and fetches data from Google Books API.}
% \item \textbf{Short Project Title.} {Build a project that does something and had quantified success using A, B, and C. This project's description spans two lines and also won an award.}
\end{rSection} 

%----------------------------------------------------------------------------------------
\begin{rSection}{Organizational Experiences}

\textbf{Core Team: Curriculum} \hfill Nov 2022 - Ongoing\\
Google Developer Student Club UNESA 
 \begin{itemize}
    \itemsep -3pt {} 
     \item Designing learning path and roadmap, and finding sources depending on learning needs.
     \item Collaborate and communicate with core team to achieve the organization's goals
 \end{itemize}

\end{rSection}


\end{document}
