\documentclass{resume} 

\usepackage[left=0.4 in,top=0.4in,right=0.4 in,bottom=0.4in]{geometry} 
\newcommand{\tab}[1]{\hspace{.2667\textwidth}\rlap{#1}} 
\newcommand{\itab}[1]{\hspace{0em}\rlap{#1}}
\name{Muhammad Rizqi Ardiansyah} 
\address{+62 812 1002 9357 \\ Surabaya, East Java, Indonesia} 
\address{\href{mailto:muhrizqiardi@gmail.com}{muhrizqiardi@gmail.com} \\ \href{https://www.linkedin.com/in/muhrizqiardi/}{linkedin.com/in/muhrizqiardi/} \\ \href{https://muhrizqiardi.vercel.app}{muhrizqiardi.vercel.app} \\ \href{https://github.com/muhrizqiardi}{github.com/muhrizqiardi}}

\begin{document}

\begin{rSection}{ABOUT}
{A front end engineer enthusiastic about technology. Fast-learning and always ready of what's to come in this rapidly-moving industry. Always take great attention to detail and eager to contribute to team's success through hard work and excellent organizational skills.}
\end{rSection}

\begin{rSection}{Education}
{\bf State University of Surabaya (UNESA)}, Informatics Engineering \hfill {2020 - Ongoing}
\end{rSection}

\begin{rSection}{SKILLS}
\begin{tabular}{ @{} >{\bfseries}l @{\hspace{6ex}} l }
Programming Languages & HTML, CSS, JavaScript, TypeScript, Python, SQL \\
Frameworks & React, Next.js, Tailwind CSS \\
Tools and Other Concepts & Git, Gitflow, MongoDB, PostgreSQL, Monorepo \\
\end{tabular}\\
\end{rSection}

\begin{rSection}{EXPERIENCE}
\textbf{Freelance Frontend Engineer} \hfill Dec 2021 - Jun 2022\\
Rumah Coding Cerdas 
\begin{itemize}
\itemsep -3pt {} 
\item Developed an internal tooling software  
\item Implement a single-page application (SPA) using React based on a provided design and requirements
\item Write a unit tests for the written code
\item Collaborate with the team to implement new features and solve issues
\end{itemize}
\end{rSection} 

\begin{rSection}{Organizational Experiences}
\textbf{Core Team: Curriculum} \hfill Nov 2022 - Ongoing\\
Google Developer Student Club UNESA 
\begin{itemize}
\itemsep -3pt {} 
\item Designing learning path and roadmap for a study jam, and finding sources depending on learning needs
\item Guiding members of study jam to follow through web development courses
\item Collaborate and communicate with core team to achieve the organization's goals
\end{itemize}
\end{rSection}

\begin{rSection}{PROJECTS}
\vspace{-1.25em}
\item \textbf{Journal App} {\hfill \href{https://journal-app-lilac.vercel.app/}{journal-app-lilac.vercel.app} | \href{https://github.com/muhrizqiardi/journal-app}{github.com/muhrizqiardi/journal-app}\\
A web application that allows users to keep a journal and track their mood. It also has a feature that shows monthly or yearly statistics of the user's mood. It was built using TypeScript, Next.js 13 with Server Components, Tailwind CSS, and built on top of Supabase's PostgreSQL instance.}
\item \textbf{LABCO} {\hfill \href{https://labco.vercel.app/}{labco.vercel.app} | \href{https://github.com/muhrizqiardi/labco}{github.com/muhrizqiardi/labco}\\
A lab management system made for final project in a Software Engineering class with my teammates. The app was made using Next.js framework, both the frontend and backend. Used Tailwind CSS for the styling, and MongoDB for the non-relational database.}
\item \textbf{SpaceWork} {\hfill \href{https://spacework.vercel.app/}{spacework.vercel.app} | \href{https://github.com/muhrizqiardi/final-project-pemr-web}{github.com/muhrizqiardi/final-project-pemr-web}\\
SpaceWork is a web application for listing remote jobs that was created for a final project in a web programming class with my team. The front-end was built using React and styled with Tailwind CSS, while the back-end was built using Fastify. The database used in the project is PostgreSQL, which is hosted on Supabase.}
\item \textbf{Booktracker} {\hfill\href{https://github.com/riosiu/kelompok6-flutter}{github.com/riosiu/kelompok6-flutter}\\
A mobile app to track book progress, made for project in Mobile Programming class with my team. The app was made using Flutter. It stores data using SQLite, and fetches data from Google Books API.}
\end{rSection} 

\end{document}
